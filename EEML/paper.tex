\documentclass[a4paper]{article}
\usepackage{amsmath,amsthm,amsfonts,amssymb} % packages for math
\usepackage{graphicx,tikz,pgfplots,setspace} % packages for graphics
\usepackage[margin=2cm]{geometry}
\PassOptionsToPackage{hyphens}{url}
\doublespacing
\usepackage{caption}
\usepackage{subcaption}
%\usepackage{gensymb}
\usepackage[inline]{enumitem}
\usepackage{amsmath}
\usepackage{hanging}
\usepackage{csvsimple}
\usepackage{alltt}
\usepackage{algorithm}
\usepackage{boxedminipage}
\usepackage[noend]{algpseudocode}
\usepackage[colorlinks=true]{hyperref} % creates hyper links for references
\usepackage[utf8]{inputenc}
\usepackage[T1]{fontenc}
\usepackage{float}
\usepackage{etoolbox}
\pgfplotsset{compat=1.16}
\usepgfplotslibrary{fillbetween}
\usetikzlibrary{backgrounds}
\usetikzlibrary{arrows,decorations.markings}
\usepackage[final]{pdfpages}
%% Bibliography package
\usepackage[
backend=biber,
style=numeric,
citestyle=numeric,
sorting=nty
]{biblatex}

\addbibresource{sources.bib} %Imports bibliography file
\usepackage[colorlinks=true]{hyperref}
\hypersetup{citecolor=red}


% ========================================================================
% Put your lab title, names, and a brief abstract here.
\title{\vspace{-6em} Newton Algorithms On Grassmann Manifolds \\  Fellowship of the Mind, Milos Vukadinovic  \vspace{-3em} }
\date{}
% ========================================================================


%\setlength{\topmargin}{30mm}
%\addtolength{\topmargin}{-1in}
%\addtolength{\topmargin}{-\headsep}
%\addtolength{\topmargin}{-\headheight}
%\addtolength{\topmargin}{-\topskip}

%\setlength{\textheight}{270mm}
%\addtolength{\textheight}{\topskip}
%\addtolength{\textheight}{-\footskip}
%\addtolength{\textheight}{-30pt}

%\setlength{\oddsidemargin}{-1in}
%\addtolength{\oddsidemargin}{20mm}
%\setlength{\evensidemargin}{\oddsidemargin}

%\setlength{\textwidth}{170mm}

\newtheorem{defn}{Definition}[section]
 \newtheorem{thm}{Theorem}[section]
 \newtheorem{Lemma}{Lemma}[section]
 \newtheorem{Claim}{Claim}[section]
 \newtheorem{Prop}{Proposition}[section]
  \theoremstyle{definition}\newtheorem{Ex}{Example}[section]
 \newtheorem{Cor}{Corollary}[section]
 \newtheorem{claim}{Claim}[section]
 \newtheorem{conj}{Conjecture}  

\usepackage {amsfonts,amssymb}
%\usepackage{mathbbol}
\usepackage{latexsym}
\usepackage{mathrsfs}
\input xy
\xyoption{all}

\newcommand {\op}{\mathcal{O}\mathfrak{p}}
\newcommand {\Def}{\textrm{Def}}
\newcommand {\MC} {\textrm{MC}}
\newcommand {\Art}{\textrm{Art}_\CC}
\newcommand {\Kur}{\textrm{Kur}}
\newcommand {\LG} {^LG}
\newcommand{\fart}{\textrm{FArt}_\CC}
\newcommand{\fun}{\textrm{Fun}}
\newcommand{\sets}{\textrm{Sets}}
\newcommand{\tops}{\textrm{Top}}

\newcommand {\BB}{\mathbb{B}}
\newcommand {\CC}{\mathbb{C}}
\newcommand {\FF}{\mathbb{F}}
\newcommand {\KK}{\mathbb{K}}
\newcommand {\MM}{\mathbb{M}}
\newcommand{\NN}{\mathbb{N}}
\newcommand {\PP}{\mathbb{P}}
\newcommand{\QQ}{\mathbb{Q}}
\newcommand {\RR}{\mathbb{R}}
\newcommand {\SSS}{\mathbb{S}}
\newcommand {\VV}{\mathbb{V}}
\newcommand {\HH}{\mathbb{H}}
\newcommand {\WW}{\mathbb{W}}
\newcommand{\YY}{\mathbb{Y}}
\newcommand{\ZZ}{\mathbb{Z}}


\newcommand {\bal}{\boldsymbol{\alpha}}
\newcommand {\bbe}{\boldsymbol{\beta}}
\newcommand {\bga}{\boldsymbol{\gamma}}
\newcommand {\bmu}{\boldsymbol{\mu}}
\newcommand {\bom}{\boldsymbol{\omega}}
\newcommand {\bth}{\boldsymbol{\theta}}
\newcommand {\bph}{\boldsymbol{\phi}}
\newcommand {\bdh}{\boldsymbol{h}}
\newcommand {\bdk}{\boldsymbol{k}}
\newcommand {\bdE}{\boldsymbol{E}}
\newcommand {\bdU}{\boldsymbol{U}}
\newcommand {\bdP}{\boldsymbol{P}}
\newcommand {\ba}{{\bf a}}
\newcommand {\bb}{{\bf b}}
\newcommand {\bc}{{\bf c}}
\newcommand {\bd}{{\bf d}}
\newcommand {\bg}{{\bf g}}
\newcommand {\be}{{\bf e}}
\newcommand {\bdf}{{\bf f}}
\newcommand {\bp}{{\bf p}}
\newcommand {\bq}{{\bf q}}
\newcommand {\bv}{{\bf v}}
\newcommand {\bh}{{\bf h}}
\newcommand {\bk}{{\bf k}}
\newcommand {\br}{{\bf r}}
\newcommand {\bdu}{{\bf u}}
\newcommand {\bdv}{{\bf v}}
\newcommand {\bi}{{\bf i}}
\newcommand {\bj}{{\bf j}}
\newcommand {\bn}{{\bf n}}
\newcommand {\bs}{{\bf s}}
\newcommand {\bt}{{\bf t}}
\newcommand {\bu}{{\bf u}}
\newcommand {\bw}{{\bf w}}
\newcommand {\bx}{{\bf x}}
\newcommand{\by}{{\bf y}}
\newcommand {\bz}{{\bf z}}
\newcommand {\bB}{{\bf B}}
\newcommand{\bD}{{\bf D}}
\newcommand {\bE}{{\bf E}}
\newcommand {\bF}{{\bf F}}
\newcommand {\bG}{{\bf G}}
\newcommand {\bH}{{\bf H}}
\newcommand {\bK}{{\bf K}}
\newcommand {\bL}{{\bf L}}
\newcommand {\bM}{{\bf M}}
\newcommand {\bN}{{\bf N}}
\newcommand {\bO}{{\bf O}}
\newcommand {\bP}{{\bf P}}
\newcommand {\bQ}{{\bf Q}}
\newcommand {\bR}{{\bf R}}
\newcommand {\bT}{{\bf T}}
\newcommand {\bS}{{\bf S}}
\newcommand {\bU}{{\bf U}}
\newcommand {\bV}{{\bf V}}
\newcommand {\bW}{{\bf W}}
\newcommand {\bgamma}{\boldsymbol\gamma}
\newcommand {\bdelta}{\boldsymbol\delta}
\newcommand {\bDelta}{\boldsymbol\Delta}
%\newcommand{\qed}{{\ \bf qed}}

\newcommand {\rroot}{\mathbf{root}}
\newcommand {\coroot}{\mathbf{coroot}}
\newcommand {\weight}{\mathbf{weight}}
\newcommand {\coweight}{\mathbf{coweight}}
 \newcommand{\higgs}{\textrm{Higgs}}
\newcommand{\bun}{\textrm{Bun}}
\newcommand{\rk}{\textrm{rk}}
\newcommand{\ext}{\textrm{Ext}}
% \newcommand {\id}{\mathbb{1}}
% Use \id if using mathbbol instead of amssymb
\newcommand{\range}{\textrm{Range}}
\newcommand{\arccot}{\textrm{arccot}}

\newcommand{\thickslash}{\mathbin{\!\!\pmb{\fatslash}}}


\newcommand{\cA}{\mathcal{A}}
\newcommand{\cB}{\mathcal{B}}
\newcommand{\cC}{\mathcal{C}}
\newcommand{\cD}{\mathcal{D}}
\newcommand{\cE}{\mathcal{E}}
\newcommand{\cF}{\mathcal{F}}
\newcommand{\cG}{\mathcal{G}}
\newcommand{\cH}{\mathcal{H}}
\newcommand{\cI}{\mathcal{I}}
\newcommand{\cJ}{\mathcal{J}}
\newcommand{\cK}{\mathcal{K}}
\newcommand{\cL}{\mathcal{L}}
\newcommand {\cM}{\mathcal{M}}
\newcommand {\cN}{\mathcal{N}}
\newcommand {\cO}{\mathcal{O}}
\newcommand{\cP}{\mathcal{P}}
\newcommand{\cQ}{\mathcal{Q}}
\newcommand{\cR}{\mathcal{R}}
\newcommand{\cS}{\mathcal{S}}
\newcommand{\cT}{\mathcal{T}}
\newcommand{\cU}{\mathcal{U}}
\newcommand{\cV}{\mathcal{V}}
\newcommand{\cW}{\mathcal{W}}
\newcommand{\cX}{\mathcal{X}}
\newcommand{\cY}{\mathcal{Y}}
\newcommand{\cZ}{\mathcal{Z}}

\newcommand{\loc}{\mathcal{L}oc}
\newcommand{\Loc}{\textrm{Loc}}
\newcommand{\cih}{\mathpzc{h}}
\newcommand{\cx}{\mathpzc{x}}
\newcommand{\cy}{\mathpzc{y}}
\newcommand{\ce}{\mathpzc{e}}
\newcommand{\cf}{\mathpzc{f}}
\newcommand{\cl}{\mathpzc{l}}




 




\newcommand{\scA}{\mathscr{A}}
\newcommand{\scB}{\mathscr{B}}
\newcommand{\scC}{\mathscr{C}}
\newcommand{\scD}{\mathscr{D}}
\newcommand{\scE}{\mathscr{E}}
\newcommand{\scF}{\mathscr{F}}
\newcommand{\scG}{\mathscr{G}}
\newcommand{\scH}{\mathscr{H}}
\newcommand{\scI}{\mathscr{I}}
\newcommand{\scJ}{\mathscr{J}}
\newcommand{\scK}{\mathscr{K}}
\newcommand{\scL}{\mathscr{L}}
\newcommand{\scM}{\mathscr{M}}
\newcommand{\scP}{\mathscr{P}}
\newcommand{\scR}{\mathscr{R}}
\newcommand{\scO}{\mathscr{O}}
\newcommand{\scS}{\mathscr{S}}
\newcommand{\scT}{\mathscr{T}}
\newcommand{\scU}{\mathscr{U}}
\newcommand{\scV}{\mathscr{V}}
\newcommand{\scW}{\mathscr{W}}
\newcommand{\scX}{\mathscr{X}}
\newcommand{\scY}{\mathscr{Y}}
\newcommand{\scZ}{\scZ}

\newcommand{\uR}{\underline{\mathbb{R}}}
\newcommand {\uC}{\underline{\mathbb{C}}}


\newcommand{\fh}{\mathfrak{h}}
\newcommand{\fa}{\mathfrak{a}}
\newcommand{\fb}{\mathfrak{b}}
\newcommand{\fc}{\mathfrak{c}}
\newcommand{\fg}{\mathfrak{g}}
\newcommand{\fk}{\mathfrak{k}}
\newcommand{\fl}{\mathfrak{l}}
\newcommand{\fm}{\mathfrak{m}}
\newcommand{\fn}{\mathfrak{n}}
\newcommand{\fo}{\mathfrak{o}}
\newcommand{\fp}{\mathfrak{p}}
\newcommand{\fr}{\mathfrak{r}}
\newcommand{\fs}{\mathfrak{s}}
\newcommand{\fsu}{\mathfrak{su}}
\newcommand{\ft}{\mathfrak{t}}
\newcommand{\slt}{\mathfrak{sl}_2(\CC)}
\newcommand{\sln}{\mathfrak{sl}(n)}
\newcommand{\fsl}{\mathfrak{sl}}
\newcommand{\fu}{\mathfrak{u}}
\newcommand{\fv}{\mathfrak{v}}
\newcommand{\fx}{\mathfrak{x}}
\newcommand{\fy}{\mathfrak{y}}
\newcommand{\fz}{\mathfrak{z}}
\newcommand{\fA}{\mathfrak{A}}
\newcommand{\fB}{\mathfrak{B}}
\newcommand{\fD}{\mathfrak{D}}
\newcommand{\fM}{\mathfrak{M}}
\newcommand{\fR}{\mathfrak{R}}
\newcommand {\fU}{\mathfrak{U}}
\newcommand {\fV}{\mathfrak{V}}
\newcommand {\fW}{\mathfrak{W}}
\newcommand{\fX}{\mathfrak{X}}
\newcommand{\faff}{\mathfrak{aff}}

\newcommand{\Aff}{\textrm{Aff}}

\newcommand{\sym}{\textrm{Sym}}

\newcommand {\dbar}{\overline{\partial}}
\newcommand {\zbar}{\overline{z}}
\newcommand {\zvec}{\underline{z}}
\newcommand {\dzbar}{d\overline{z}}
\newcommand {\Nbar}{\overline{N}}
\newcommand {\Kbar}{\overline{K}}
\newcommand{\diff}{\textrm{Diff}}

%\newcommand {\hom}{\textrm{Hom}}

\newcommand{\mhom}{\textrm{Hom}}
\newcommand {\mend}{\textrm{End}}
\newcommand {\misom}{\textrm{Isom}}
\newcommand {\maut}{\textrm{Aut}}
\newcommand{\pr}{\textrm{pr}}

\newcommand {\sisom}{\underline{Isom}}
\newcommand {\saut}{\underline{Aut}}
\newcommand {\shom}{\textrm{\underline{Hom}}}
\newcommand {\send}{\underline{End} }

\newcommand{\dra}{M^{an}_{DR}(X,G)}
\newcommand{\dr}{M_{DR}(X,G)}

\newcommand {\ad}{\textrm{ad} }
\newcommand{\Ad}{\textrm{Ad}}

\newcommand{\lspan}{\textrm{span}}
\newcommand{\img}{\textrm{Im }}
\newcommand{\spec}{\textrm{Spec }}
\newcommand{\specan}{\textrm{Spec}^{an}}
\newcommand{\gspec}{\underline{\textrm{Spec }}}
\newcommand {\cok}{\textrm{coker}}
\newcommand{\tot}{\textrm{tot }}
\newcommand{\tildel}{\widetilde{\delta}}
\newcommand{\ctimes}{\otimes_\CC}
\newcommand{\sotimes}{\otimes_{\cO_X}}
\newcommand{\pic}{\textrm{Pic}}
\newcommand{\tr}{\textrm{tr }}

\newcommand  {\eps}{\varepsilon}
\newcommand {\kap}{\varkappa}
\newcommand {\io}{\iota}
\newcommand {\fii}{\varphi}

\newcommand{\Higgs}{{\bf Higgs}}
\newcommand{\Bun}{{\bf Bun}}
\newcommand{\gHiggs}{\op{\boldsymbol{\mathcal{H}iggs}}}
\newcommand{\Prym}{{\bf Prym}}
\newcommand{\Jac}{{\bf Jac}}
%\newcommand{\bh}{\boldsymbol{h}}
%\newcommand{\bH}{\boldsymbol{\mathcal{H}}}
\newcommand{\rts}{{\sf root}}
\newcommand{\wts}{{\sf weight}}
\newcommand{\crts}{{\sf coroot}}
\newcommand{\cwts}{{\sf coweight}}
\newcommand{\chr}{{\sf char}}
\newcommand{\cchr}{{\sf cochar}}

\newcommand{\Aut}{\textrm{Aut}}
\newcommand{\Der}{\textrm{Der}}
\newcommand{\spin}{\textrm{Spin}}
\newcommand{\spinc}{\textrm{Spin}^c}
%\newcommand{\U}{\boldsymbol{U(1)}}

\newcommand{\Mat}{\textrm{Mat}}

\newcommand{\hookr}{\hookrightarrow}

%%%%%%%%%%%%%%%%%%%%%%%%%%%%%%%%%%%%%%%%%%%%%%%%%%%%%%%%%%%%%%%%%%%%%%%%%
% Long exact sequence macro
%%%%%%%%%%%%%%%%%%%%%%%%%%%%%%%%%%%%%%%%%%%%%%%%%%%%%%%%%%%%%%%%%%%%%%%%%

\newcommand{\les}[9]{
\xymatrix{
 0 \ar[r] & {#1} \ar[r]  &  {#2} \ar[r]  &  {#3}
\ar@{->}`r/10pt[d] `[l] `^dl[dlll]  `^dr/10pt[dll]    [dll] \\
 &  {#4} \ar[r] & {#5} \ar[r] & {#6}
\ar@{->}`r/10pt[d] `[l] `^dl[dlll]  `^dr/10pt[dll]    [dll] \\
 & {#7} \ar[r]  & {#8} \ar[r] & {#9}
\ar@{->}`r/10pt[d] `[l] `^dl[dlll]  `^dr/10pt[dll]    [dll] \\
 & 0 \ar[r] & \cdots & }
}


%%%%%%%%%%%%%%%%%%%%%%%%%%%%%%%%%%%%%%%%%%%%%%%%%%%%%%%%%%%%%%%%%%%%%%%%%

\newcommand{\lestwo}[9]{
\xymatrix{     
 0 \ar[r] & {#1} \ar[r]  &  {#2} \ar[r]  &  {#3} 
\ar@{->}`r/10pt[d] `[l] `^dl[dlll]  `^dr/10pt[dll]    [dll] \\
 &  {#4} \ar[r] & {#5} \ar[r] & {#6} 
\ar@{->}`r/10pt[d] `[l] `^dl[dlll]  `^dr/10pt[dll]    [dll] \\
 & {#7} \ar[r]  & {#8} \ar[r] & {#9} }
}

%%%%%%%%%%%%%%%%%%%%%%%%%%%%%%%%%%%%%%%%%%%%%%%%%%%%%%%%%%%%%%%%%%%%%%%%%
% Long exact sequence macro
%%%%%%%%%%%%%%%%%%%%%%%%%%%%%%%%%%%%%%%%%%%%%%%%%%%%%%%%%%%%%%%%%%%%%%%%%

\newcommand{\lesthree}[5]{
\xymatrix{     
 0 \ar[r] & {#1} \ar[r]  &  {#2} \ar[r]  &  {#3} 
\ar@{->}`r/10pt[d] `[l] `^dl[dlll]  `^dr/10pt[dll]    [dll] \\
 &  {#4} \ar[r] & {#5} & }
}


%%%%%%%%%%%%%%%%%%%%%%%%%%%%%%%%%%%%%%%%%%%%%%%%%%%%%%%%%%%%%%%%%%%%%%%%%
% Long exact sequence macro
%%%%%%%%%%%%%%%%%%%%%%%%%%%%%%%%%%%%%%%%%%%%%%%%%%%%%%%%%%%%%%%%%%%%%%%%%

\newcommand{\lesfour}[8]{
\xymatrix{     
 0 \ar[r] & {#1} \ar[r]  &  {#2} \ar[r]  &  {#3} 
\ar@{->}`r/10pt[d] `[l] `^dl[dlll]  `^dr/10pt[dll]    [dll] \\
 &  {#4} \ar[r]^-{#8} & {#5} \ar[r] & {#6} 
\ar@{->}`r/10pt[d] `[l] `^dl[dlll]  `^dr/10pt[dll]    [dll] \\
 & {#7} \ar[r]  & \cdots  &  }
}

\newcommand{\equivclass}[1]{%
  #1/{\sim}%
}
\newcommand{\equivcls}[1]{%
  #1/\!{\sim}%
}

%


\begin{document}
\maketitle % This line creates the title (DO NOT CHANGE)
\section{Introduction}
We will present Newton Methods on the functions whose domain lies on the Grassmann Manifolds. 
Such functions are ubiquitous because data with subspace-structured
features, orthogonality, or low-rank constraints can be
naturally expressed using the Grassmann manifold.
We consider two different representations of the
Grassmann manifold and define Newton's method for each of them.
By exploiting the underlying geometry of data, Grassmannian algorithms
converge faster, generalize better and perform as well as the best-known 
special (problem-specific) algorithms. 
\section{Representations of Grassmann Manifold}
The Grassmannian $G(p,n)$ is the set of linear subspaces of dimension $p$
in the real coordinate space of dimension $n$. Naturally it can be expressed as $G_0(p,n) = \equivclass{\{ A \in \RR^{n \times p} \; | \; rk A = p \}} $
Where two elements $A$ and $B$ are equivalent if there exists an invertible $p \times p$ matrix s.t. $B=Ag$. We get a manifold structure on $G_0(p,n)$ by finding charts, atlas and transition maps.
We will consider two other representations of the Grassmann manifold and develop algorithms for them. We will use subscript labeling to distinguish between them.
First, as a Riemannian quotient manifold of compact Stiefel manifold $St(p,n)$.
$$
 St(p,n) = \{ X \in \RR^{n \times p} : X^T X = I_p \} 
$$
\begin{equation} \label{quot}
 G_1(p,n) = \equivclass{St(p,n)}, \quad  Y \sim X \text{ if } \exists \; Q \in O(p), \text{ s.t. } Y=XQ 
\end{equation}
To obtain a manifold structure on \ref{quot}, consider the following setup:
$$ O(n) \stackrel{\psi}{\to}  St(n,p) \stackrel{\phi}{\to} G(p,n), \quad \psi(Q) = (:,1:p), \quad \phi(X) = X X^T$$
Second, we identify Grassmann manifold with a set of self-adjoint projection operators.
\begin{equation}\label{proj}
G_{2}(p,n) = \{ M \in Mat_{n \times n} \; | \; M^T = M, M^2 = M, TrM=p \}
\end{equation}
We obtain a manifold structure on $G_{2}(p,n)$ by considering a map $h: G_0(p,n) \to G_2(p,n), \quad h(A) = A(A^T A^{-1}) A^T$ 
\section{Algorithms}
We perform optimization with Newton's root-finding method by applying it to the derivative of the twice differentiable function $f$ to find the critical points.
Our objective is to minimize $F: G(p,n) \to \RR$. \newline
For \ref{quot} we have $F: G_1(p,n) \to \RR $, $ \; F(Y) = F(YQ), \; Y \in G_1(p,n), \; Q \in O(p), Y^T Y = I_p$
\newline
$F_Y$ is a matrix of partial derivatives of $F$ with respect to $Y$. 
$(F_{Y})_{ij} = \frac{\delta F}{\delta Y_{i,j}} $, $ F_{YY} = \frac{\delta^2 F}{\delta Y_{i,j}\delta Y_{k,l}}$. 
\newline
Delta is in a tangent space $\Delta \in T_{Y} G(p,n)$,
$ \; Hess F(\Delta_1, \Delta_2) = F_{YY}(\Delta_1, \Delta_2) - tr(\Delta_1^T \Delta_2 Y^T F_Y) $
\newline
\begin{algorithm}
\caption{Newton's method for minimizing $F(Y)$ on $G_1(p,n)$ }\label{alg:quotAlg}
\begin{algorithmic}[1]
\State \textcolor{red}{// Input: $F(\cdot)$ and the initial choice of $Y$ such that $Y^T Y = I_p$}
\State \textcolor{red}{// Output: $Y$ for which $F(Y)$ gives the minimum value}
\Procedure{MINIMIZE}{}
\While{$numSteps--$}\Comment{We have to choose the number of steps, or define some stopping criteria}
\State $G = F_Y - Y Y^T F_Y$
\State $\Delta = -Hess^{-1} G$ such that $Y^T \Delta = 0$ and $F_{YY}(\Delta) - \Delta(Y^T F_Y) = -G$
\State
\State Move from $Y$ in the direction $\Delta$ to $Y(1)$ using the formula 
\State $Y(t) = Y V \cos(\Sigma t) V^T + U \sin(\sigma t) V^T$ \Comment{$U \Sigma V^T$ is SVD of $\Delta$}
\EndWhile
\State \textbf{return} $Y$
\EndProcedure
\end{algorithmic}
\end{algorithm}

For \ref{proj} we define $F: Sym(n) \to \RR $ and  $f: G_2(p,n) \to \RR$ s.t. $f= F|_{G(p,n)}. \quad$
$ ad_p(X) = [P, X] = PX -  XP $
$M_Q$ subscript means that we are taking only $Q$ part from the $QR$ decomposition of $M$
\begin{algorithm}
\caption{Newton's method for minimizing $F(M)$ on $G_2(p,n)$ }\label{alg:projAlg}
\begin{algorithmic}[1]
\State \textcolor{red}{// Input: $F(\cdot)$ and the initial choice of $M$ such that $M^T=M, M^2=M, TrM=p$}
\State \textcolor{red}{// Output: $M$ for which $F(M)$ gives the minimum value}
\Procedure{MINIMIZE}{}
\While{$numSteps--$}\Comment{We have to choose the number of steps, or define some stopping criteria}
\State Solve
\State $ad^2_{M} Hess_{F}(M)(ad_{M}\Omega) - ad_{M} ad_{\nabla F (M) } ad_{M} \Omega = -ad^2_{M} \nabla_{F}(M)$
\State for $\Omega \in skew_-sym(n) $
\State
\State Solve 
\State $M = \Theta^T \begin{bmatrix} I_m & 0 \\ 0 & 0 \end{bmatrix} \Theta$ \Comment{$\Theta$ is orthonormal}
\State for $ \Theta \in SO_n$
\State
\State $ M = \Theta^T (\Theta( I - ad^2_{M} \Omega ) \Theta^T)_Q     \Theta M \Theta^T    (\Theta (I - ad^2_{M} \Omega) \Theta^T)_Q^T    \Theta$
\EndWhile
\State \textbf{return} $M$
\EndProcedure
\end{algorithmic}
\end{algorithm}
\section{Results}
We adapt both algorithms to compute solutions for the following problems: Rayleigh quotient, Dominant Invariant Subspace, and CS decomposition.
Our algorithms have the potential for better computational complexity than the best-known algorithms for mentioned problems.
\section{Discussion}
Our contribution is the rigorous mathematical derivation, implementation and fine-tuning of the Grassmannian Newton algorithms \cite{edelman_geometry_1998} \cite{helmke_newtons_2007}.
The author's goal is to define optimization methods that can be used for learning on Grassmannians.
Once we define neural networks on Grassmannians we will be able to solve subspace learning problems like:  visual domain adaptation and video emotion classification more efficiently \cite{huang_building_2018} \cite{zhang_grassmannian_2018} .
\newpage
\section{Sources}
\printbibliography[heading=none]
\end{document}
