\documentclass[11pt,a4paper]{report}
\usepackage{graphicx}
\graphicspath{{./images/}}
\usepackage[utf8]{inputenc}
\usepackage{amsmath}
%\usepackage{tabto}
\usepackage{amsfonts}
\usepackage{amssymb}
\usepackage{tikz}
\usepackage{amsthm}
\usepackage{xcolor}
\usepackage{setspace}
\usepackage[margin=1in]{geometry}

\usepackage[colorlinks=true,          % link colors, set to 'false' for print version
            linkcolor=blue,
            citecolor=red,
            urlcolor=blue]{hyperref}
\onehalfspacing            

%\setlength{\topmargin}{30mm}
%\addtolength{\topmargin}{-1in}
%\addtolength{\topmargin}{-\headsep}
%\addtolength{\topmargin}{-\headheight}
%\addtolength{\topmargin}{-\topskip}

%\setlength{\textheight}{270mm}
%\addtolength{\textheight}{\topskip}
%\addtolength{\textheight}{-\footskip}
%\addtolength{\textheight}{-30pt}

%\setlength{\oddsidemargin}{-1in}
%\addtolength{\oddsidemargin}{20mm}
%\setlength{\evensidemargin}{\oddsidemargin}

%\setlength{\textwidth}{170mm}

\newtheorem{defn}{Definition}[section]
 \newtheorem{thm}{Theorem}[section]
 \newtheorem{Lemma}{Lemma}[section]
 \newtheorem{Claim}{Claim}[section]
 \newtheorem{Prop}{Proposition}[section]
  \theoremstyle{definition}\newtheorem{Ex}{Example}[section]
 \newtheorem{Cor}{Corollary}[section]
 \newtheorem{claim}{Claim}[section]
 \newtheorem{conj}{Conjecture}  

\usepackage {amsfonts,amssymb}
%\usepackage{mathbbol}
\usepackage{latexsym}
\usepackage{mathrsfs}
\input xy
\xyoption{all}

\newcommand {\op}{\mathcal{O}\mathfrak{p}}
\newcommand {\Def}{\textrm{Def}}
\newcommand {\MC} {\textrm{MC}}
\newcommand {\Art}{\textrm{Art}_\CC}
\newcommand {\Kur}{\textrm{Kur}}
\newcommand {\LG} {^LG}
\newcommand{\fart}{\textrm{FArt}_\CC}
\newcommand{\fun}{\textrm{Fun}}
\newcommand{\sets}{\textrm{Sets}}
\newcommand{\tops}{\textrm{Top}}

\newcommand {\BB}{\mathbb{B}}
\newcommand {\CC}{\mathbb{C}}
\newcommand {\FF}{\mathbb{F}}
\newcommand {\KK}{\mathbb{K}}
\newcommand {\MM}{\mathbb{M}}
\newcommand{\NN}{\mathbb{N}}
\newcommand {\PP}{\mathbb{P}}
\newcommand{\QQ}{\mathbb{Q}}
\newcommand {\RR}{\mathbb{R}}
\newcommand {\SSS}{\mathbb{S}}
\newcommand {\VV}{\mathbb{V}}
\newcommand {\HH}{\mathbb{H}}
\newcommand {\WW}{\mathbb{W}}
\newcommand{\YY}{\mathbb{Y}}
\newcommand{\ZZ}{\mathbb{Z}}


\newcommand {\bal}{\boldsymbol{\alpha}}
\newcommand {\bbe}{\boldsymbol{\beta}}
\newcommand {\bga}{\boldsymbol{\gamma}}
\newcommand {\bmu}{\boldsymbol{\mu}}
\newcommand {\bom}{\boldsymbol{\omega}}
\newcommand {\bth}{\boldsymbol{\theta}}
\newcommand {\bph}{\boldsymbol{\phi}}
\newcommand {\bdh}{\boldsymbol{h}}
\newcommand {\bdk}{\boldsymbol{k}}
\newcommand {\bdE}{\boldsymbol{E}}
\newcommand {\bdU}{\boldsymbol{U}}
\newcommand {\bdP}{\boldsymbol{P}}
\newcommand {\ba}{{\bf a}}
\newcommand {\bb}{{\bf b}}
\newcommand {\bc}{{\bf c}}
\newcommand {\bd}{{\bf d}}
\newcommand {\bg}{{\bf g}}
\newcommand {\be}{{\bf e}}
\newcommand {\bdf}{{\bf f}}
\newcommand {\bp}{{\bf p}}
\newcommand {\bq}{{\bf q}}
\newcommand {\bv}{{\bf v}}
\newcommand {\bh}{{\bf h}}
\newcommand {\bk}{{\bf k}}
\newcommand {\br}{{\bf r}}
\newcommand {\bdu}{{\bf u}}
\newcommand {\bdv}{{\bf v}}
\newcommand {\bi}{{\bf i}}
\newcommand {\bj}{{\bf j}}
\newcommand {\bn}{{\bf n}}
\newcommand {\bs}{{\bf s}}
\newcommand {\bt}{{\bf t}}
\newcommand {\bu}{{\bf u}}
\newcommand {\bw}{{\bf w}}
\newcommand {\bx}{{\bf x}}
\newcommand{\by}{{\bf y}}
\newcommand {\bz}{{\bf z}}
\newcommand {\bB}{{\bf B}}
\newcommand{\bD}{{\bf D}}
\newcommand {\bE}{{\bf E}}
\newcommand {\bF}{{\bf F}}
\newcommand {\bG}{{\bf G}}
\newcommand {\bH}{{\bf H}}
\newcommand {\bK}{{\bf K}}
\newcommand {\bL}{{\bf L}}
\newcommand {\bM}{{\bf M}}
\newcommand {\bN}{{\bf N}}
\newcommand {\bO}{{\bf O}}
\newcommand {\bP}{{\bf P}}
\newcommand {\bQ}{{\bf Q}}
\newcommand {\bR}{{\bf R}}
\newcommand {\bT}{{\bf T}}
\newcommand {\bS}{{\bf S}}
\newcommand {\bU}{{\bf U}}
\newcommand {\bV}{{\bf V}}
\newcommand {\bW}{{\bf W}}
\newcommand {\bgamma}{\boldsymbol\gamma}
\newcommand {\bdelta}{\boldsymbol\delta}
\newcommand {\bDelta}{\boldsymbol\Delta}
%\newcommand{\qed}{{\ \bf qed}}

\newcommand {\rroot}{\mathbf{root}}
\newcommand {\coroot}{\mathbf{coroot}}
\newcommand {\weight}{\mathbf{weight}}
\newcommand {\coweight}{\mathbf{coweight}}
 \newcommand{\higgs}{\textrm{Higgs}}
\newcommand{\bun}{\textrm{Bun}}
\newcommand{\rk}{\textrm{rk}}
\newcommand{\ext}{\textrm{Ext}}
% \newcommand {\id}{\mathbb{1}}
% Use \id if using mathbbol instead of amssymb
\newcommand{\range}{\textrm{Range}}
\newcommand{\arccot}{\textrm{arccot}}

\newcommand{\thickslash}{\mathbin{\!\!\pmb{\fatslash}}}


\newcommand{\cA}{\mathcal{A}}
\newcommand{\cB}{\mathcal{B}}
\newcommand{\cC}{\mathcal{C}}
\newcommand{\cD}{\mathcal{D}}
\newcommand{\cE}{\mathcal{E}}
\newcommand{\cF}{\mathcal{F}}
\newcommand{\cG}{\mathcal{G}}
\newcommand{\cH}{\mathcal{H}}
\newcommand{\cI}{\mathcal{I}}
\newcommand{\cJ}{\mathcal{J}}
\newcommand{\cK}{\mathcal{K}}
\newcommand{\cL}{\mathcal{L}}
\newcommand {\cM}{\mathcal{M}}
\newcommand {\cN}{\mathcal{N}}
\newcommand {\cO}{\mathcal{O}}
\newcommand{\cP}{\mathcal{P}}
\newcommand{\cQ}{\mathcal{Q}}
\newcommand{\cR}{\mathcal{R}}
\newcommand{\cS}{\mathcal{S}}
\newcommand{\cT}{\mathcal{T}}
\newcommand{\cU}{\mathcal{U}}
\newcommand{\cV}{\mathcal{V}}
\newcommand{\cW}{\mathcal{W}}
\newcommand{\cX}{\mathcal{X}}
\newcommand{\cY}{\mathcal{Y}}
\newcommand{\cZ}{\mathcal{Z}}

\newcommand{\loc}{\mathcal{L}oc}
\newcommand{\Loc}{\textrm{Loc}}
\newcommand{\cih}{\mathpzc{h}}
\newcommand{\cx}{\mathpzc{x}}
\newcommand{\cy}{\mathpzc{y}}
\newcommand{\ce}{\mathpzc{e}}
\newcommand{\cf}{\mathpzc{f}}
\newcommand{\cl}{\mathpzc{l}}




 




\newcommand{\scA}{\mathscr{A}}
\newcommand{\scB}{\mathscr{B}}
\newcommand{\scC}{\mathscr{C}}
\newcommand{\scD}{\mathscr{D}}
\newcommand{\scE}{\mathscr{E}}
\newcommand{\scF}{\mathscr{F}}
\newcommand{\scG}{\mathscr{G}}
\newcommand{\scH}{\mathscr{H}}
\newcommand{\scI}{\mathscr{I}}
\newcommand{\scJ}{\mathscr{J}}
\newcommand{\scK}{\mathscr{K}}
\newcommand{\scL}{\mathscr{L}}
\newcommand{\scM}{\mathscr{M}}
\newcommand{\scP}{\mathscr{P}}
\newcommand{\scR}{\mathscr{R}}
\newcommand{\scO}{\mathscr{O}}
\newcommand{\scS}{\mathscr{S}}
\newcommand{\scT}{\mathscr{T}}
\newcommand{\scU}{\mathscr{U}}
\newcommand{\scV}{\mathscr{V}}
\newcommand{\scW}{\mathscr{W}}
\newcommand{\scX}{\mathscr{X}}
\newcommand{\scY}{\mathscr{Y}}
\newcommand{\scZ}{\scZ}

\newcommand{\uR}{\underline{\mathbb{R}}}
\newcommand {\uC}{\underline{\mathbb{C}}}


\newcommand{\fh}{\mathfrak{h}}
\newcommand{\fa}{\mathfrak{a}}
\newcommand{\fb}{\mathfrak{b}}
\newcommand{\fc}{\mathfrak{c}}
\newcommand{\fg}{\mathfrak{g}}
\newcommand{\fk}{\mathfrak{k}}
\newcommand{\fl}{\mathfrak{l}}
\newcommand{\fm}{\mathfrak{m}}
\newcommand{\fn}{\mathfrak{n}}
\newcommand{\fo}{\mathfrak{o}}
\newcommand{\fp}{\mathfrak{p}}
\newcommand{\fr}{\mathfrak{r}}
\newcommand{\fs}{\mathfrak{s}}
\newcommand{\fsu}{\mathfrak{su}}
\newcommand{\ft}{\mathfrak{t}}
\newcommand{\slt}{\mathfrak{sl}_2(\CC)}
\newcommand{\sln}{\mathfrak{sl}(n)}
\newcommand{\fsl}{\mathfrak{sl}}
\newcommand{\fu}{\mathfrak{u}}
\newcommand{\fv}{\mathfrak{v}}
\newcommand{\fx}{\mathfrak{x}}
\newcommand{\fy}{\mathfrak{y}}
\newcommand{\fz}{\mathfrak{z}}
\newcommand{\fA}{\mathfrak{A}}
\newcommand{\fB}{\mathfrak{B}}
\newcommand{\fD}{\mathfrak{D}}
\newcommand{\fM}{\mathfrak{M}}
\newcommand{\fR}{\mathfrak{R}}
\newcommand {\fU}{\mathfrak{U}}
\newcommand {\fV}{\mathfrak{V}}
\newcommand {\fW}{\mathfrak{W}}
\newcommand{\fX}{\mathfrak{X}}
\newcommand{\faff}{\mathfrak{aff}}

\newcommand{\Aff}{\textrm{Aff}}

\newcommand{\sym}{\textrm{Sym}}

\newcommand {\dbar}{\overline{\partial}}
\newcommand {\zbar}{\overline{z}}
\newcommand {\zvec}{\underline{z}}
\newcommand {\dzbar}{d\overline{z}}
\newcommand {\Nbar}{\overline{N}}
\newcommand {\Kbar}{\overline{K}}
\newcommand{\diff}{\textrm{Diff}}

%\newcommand {\hom}{\textrm{Hom}}

\newcommand{\mhom}{\textrm{Hom}}
\newcommand {\mend}{\textrm{End}}
\newcommand {\misom}{\textrm{Isom}}
\newcommand {\maut}{\textrm{Aut}}
\newcommand{\pr}{\textrm{pr}}

\newcommand {\sisom}{\underline{Isom}}
\newcommand {\saut}{\underline{Aut}}
\newcommand {\shom}{\textrm{\underline{Hom}}}
\newcommand {\send}{\underline{End} }

\newcommand{\dra}{M^{an}_{DR}(X,G)}
\newcommand{\dr}{M_{DR}(X,G)}

\newcommand {\ad}{\textrm{ad} }
\newcommand{\Ad}{\textrm{Ad}}

\newcommand{\lspan}{\textrm{span}}
\newcommand{\img}{\textrm{Im }}
\newcommand{\spec}{\textrm{Spec }}
\newcommand{\specan}{\textrm{Spec}^{an}}
\newcommand{\gspec}{\underline{\textrm{Spec }}}
\newcommand {\cok}{\textrm{coker}}
\newcommand{\tot}{\textrm{tot }}
\newcommand{\tildel}{\widetilde{\delta}}
\newcommand{\ctimes}{\otimes_\CC}
\newcommand{\sotimes}{\otimes_{\cO_X}}
\newcommand{\pic}{\textrm{Pic}}
\newcommand{\tr}{\textrm{tr }}

\newcommand  {\eps}{\varepsilon}
\newcommand {\kap}{\varkappa}
\newcommand {\io}{\iota}
\newcommand {\fii}{\varphi}

\newcommand{\Higgs}{{\bf Higgs}}
\newcommand{\Bun}{{\bf Bun}}
\newcommand{\gHiggs}{\op{\boldsymbol{\mathcal{H}iggs}}}
\newcommand{\Prym}{{\bf Prym}}
\newcommand{\Jac}{{\bf Jac}}
%\newcommand{\bh}{\boldsymbol{h}}
%\newcommand{\bH}{\boldsymbol{\mathcal{H}}}
\newcommand{\rts}{{\sf root}}
\newcommand{\wts}{{\sf weight}}
\newcommand{\crts}{{\sf coroot}}
\newcommand{\cwts}{{\sf coweight}}
\newcommand{\chr}{{\sf char}}
\newcommand{\cchr}{{\sf cochar}}

\newcommand{\Aut}{\textrm{Aut}}
\newcommand{\Der}{\textrm{Der}}
\newcommand{\spin}{\textrm{Spin}}
\newcommand{\spinc}{\textrm{Spin}^c}
%\newcommand{\U}{\boldsymbol{U(1)}}

\newcommand{\Mat}{\textrm{Mat}}

\newcommand{\hookr}{\hookrightarrow}

%%%%%%%%%%%%%%%%%%%%%%%%%%%%%%%%%%%%%%%%%%%%%%%%%%%%%%%%%%%%%%%%%%%%%%%%%
% Long exact sequence macro
%%%%%%%%%%%%%%%%%%%%%%%%%%%%%%%%%%%%%%%%%%%%%%%%%%%%%%%%%%%%%%%%%%%%%%%%%

\newcommand{\les}[9]{
\xymatrix{
 0 \ar[r] & {#1} \ar[r]  &  {#2} \ar[r]  &  {#3}
\ar@{->}`r/10pt[d] `[l] `^dl[dlll]  `^dr/10pt[dll]    [dll] \\
 &  {#4} \ar[r] & {#5} \ar[r] & {#6}
\ar@{->}`r/10pt[d] `[l] `^dl[dlll]  `^dr/10pt[dll]    [dll] \\
 & {#7} \ar[r]  & {#8} \ar[r] & {#9}
\ar@{->}`r/10pt[d] `[l] `^dl[dlll]  `^dr/10pt[dll]    [dll] \\
 & 0 \ar[r] & \cdots & }
}


%%%%%%%%%%%%%%%%%%%%%%%%%%%%%%%%%%%%%%%%%%%%%%%%%%%%%%%%%%%%%%%%%%%%%%%%%

\newcommand{\lestwo}[9]{
\xymatrix{     
 0 \ar[r] & {#1} \ar[r]  &  {#2} \ar[r]  &  {#3} 
\ar@{->}`r/10pt[d] `[l] `^dl[dlll]  `^dr/10pt[dll]    [dll] \\
 &  {#4} \ar[r] & {#5} \ar[r] & {#6} 
\ar@{->}`r/10pt[d] `[l] `^dl[dlll]  `^dr/10pt[dll]    [dll] \\
 & {#7} \ar[r]  & {#8} \ar[r] & {#9} }
}

%%%%%%%%%%%%%%%%%%%%%%%%%%%%%%%%%%%%%%%%%%%%%%%%%%%%%%%%%%%%%%%%%%%%%%%%%
% Long exact sequence macro
%%%%%%%%%%%%%%%%%%%%%%%%%%%%%%%%%%%%%%%%%%%%%%%%%%%%%%%%%%%%%%%%%%%%%%%%%

\newcommand{\lesthree}[5]{
\xymatrix{     
 0 \ar[r] & {#1} \ar[r]  &  {#2} \ar[r]  &  {#3} 
\ar@{->}`r/10pt[d] `[l] `^dl[dlll]  `^dr/10pt[dll]    [dll] \\
 &  {#4} \ar[r] & {#5} & }
}


%%%%%%%%%%%%%%%%%%%%%%%%%%%%%%%%%%%%%%%%%%%%%%%%%%%%%%%%%%%%%%%%%%%%%%%%%
% Long exact sequence macro
%%%%%%%%%%%%%%%%%%%%%%%%%%%%%%%%%%%%%%%%%%%%%%%%%%%%%%%%%%%%%%%%%%%%%%%%%

\newcommand{\lesfour}[8]{
\xymatrix{     
 0 \ar[r] & {#1} \ar[r]  &  {#2} \ar[r]  &  {#3} 
\ar@{->}`r/10pt[d] `[l] `^dl[dlll]  `^dr/10pt[dll]    [dll] \\
 &  {#4} \ar[r]^-{#8} & {#5} \ar[r] & {#6} 
\ar@{->}`r/10pt[d] `[l] `^dl[dlll]  `^dr/10pt[dll]    [dll] \\
 & {#7} \ar[r]  & \cdots  &  }
}

\newcommand{\equivclass}[1]{%
  #1/{\sim}%
}
\newcommand{\equivcls}[1]{%
  #1/\!{\sim}%
}

%


\include{biblio}
\author{Milos Vukadinovic}
\title{Sphere is a smooth manifold}
\begin{document}

Let $X \subseteq Mat_{3 \times 3}(\RR)$ be the set
$$ X = \{ M \mid M^T = M, \: M^2 = M,\: Tr M = 1 \}$$
We will describe a bijection between $X$ and $\RR \PP^2$.
The above definition tells us that $X$ consists of matrices that are symmetric, idempotent and whose eigenvalues add up to one.
Spectral theorem tells us that a real symmetric matrix is diagonizable.
We can also show that the eigenvectors of symmetric matrices, with distinct eigenvalues, are orthogonal.
Indeed, let $x$ and $y$ be eigenvectors of a symmetric matrix $M$, with eigenvalues $\lambda$ and $\mu$, $\lambda \neq \mu$:
$$ \lambda \langle x,y \rangle = \langle \lambda x,  y \rangle = \langle M x, y \rangle = \langle x M^T, y \rangle = \langle x, M y \rangle = \langle x, \mu y \rangle = \mu \langle x, y \rangle  $$
Therefore $(\lambda - \mu) \langle x , y \rangle = 0$, since $\lambda$ and $\mu$ are distinct $xy$=0, thus orthogonal.
Next, we can show that the eigenvalues of $M$ can be only $0$ and $1$.
Let $v$ be an eigenvector, of eigenvalue $\lambda$.
$$ M v = \lambda v $$
$$ M^2 v = M (\lambda v) = \lambda M (v) = \lambda^2 v $$
As $\bar{v} \neq 0$ $(\lambda^2 - \lambda) v = 0 \iff \lambda^2 - \lambda = 0$.
Then solving for $\lambda^2-\lambda = 0$, we get that $\lambda$ can only be $0$ or $1$.
Finally, the fact that $Tr(M)=1$ tells us that eigenvalues of $M$ are $0$,$0$ and $1$.
Therefore $\dim \ker M = 2$ and $ \dim Im(M) = 1 $. This is telling us that there is a whole plane,
that is sent to zero vector by $M$, and all vectors in the image are sitting on a line.
\newline
Thus, applying a matrix operator $M$ to a vector,
is equivalent to projecting a vector to a line in $R^3$.
So $M :R^3 \to R^3$ is the operator of orthogonal projection on the line $Im(M)$.
\newline
Now, let's explicitly define a map $\phi$, which to given line in $R^3$ assigns a corresponding matrix operator,
that will orthogonally project all the vectors in $R^3$ to that line.
$$ \phi: \RR \PP ^2 \to Mat_{3 \times 3}  \quad \phi([x:y:z]) \to A $$
To explicitly find $A$, note that we first need to find a unit vector along a line $[x:y:z] \in \RR \PP ^2$,
we can do that by normalizing coordinates. $n = \frac{1}{\sqrt{x^2 + y^2 + z^2}} [x,y,z]^T $.
Finally, to orthogonally project any $v \in R^3$ along $n$, we apply $(v n) n = v \; n \otimes n$. We can then define $\phi$ as follows:
$$ \phi([x:y:z])  = \frac{1}{\sqrt{x^2+y^2+z^2}^2} [x,y,z]^T \otimes [x,y,z] = \frac{1}{x^2 + y^2 + z^2} 
\begin{pmatrix}
x^2 & xy & xz \\
xy & y^2 & yz  \\
xz & yz & z^2
\end{pmatrix} $$
Now, we redefine $\phi: \RR \PP ^2 \to \RR^6$, because  $Mat_{3 \times 3} \supseteq Sym_{3 \times 3} \simeq \RR^6$.
\begin{defn}{Immersion}
Let $X$ and $Y$ be smooth manifolds, $\dim X =n $, $\dim Y = k$. Let $f : X \to Y$ be a smooth map.
We say that $f$ is a
\begin{itemize}
    \item submersion, if $df_p$ is surjective $\forall p \in X$
    \item immersion, if $df_p$ is injective $\forall p in X$ equivalently if $rank D_p f = \dim M, M=f(X)$
\end{itemize}
\end{defn}
\begin{defn}
    Let $f: X \to Y$ be a smooth map of smooth manifolds. We say that $f$ is an embedding if 
    \begin{itemize}
        \item f is an injective immersion
        \item X is homeomorphic to $f(X) \subset Y$ (equipped with the subspace topology)
    \end{itemize}
\end{defn}
Next, we argue that $\phi \RR \PP ^2 \to R^6$ is an embedding ($R^6$ because we are taking only non-symmetric lower triangular entries)
Namely, we will show that the following function is an embedding.
$$ \phi([x:y:z]) = \frac{1}{x^2+y^2+z^2} (x^2,xy, xz, y^2, yz, z^2) $$
First, we show that $\phi$ is well defined. Take two vectors $a$,$b \in [x:y:z]$ on the same line.
If $a$ is given by $a=[a_1,a_2,a_3]$ then $b = [k a_1, k a_2, k a_3]$ for $k \in \RR$. We need to show that 
$\phi([a]) = \phi([b])$.
$$\phi([a_1,a_2,a_3]) = \frac{(a_1^2, a_1a_2, a_2^2, a_2a_3, a_3^2)}{a_1^2+a_2^2+a_3^2}$$
$$\phi([ka,kb,kc]) = 
\frac{(k^2 a_1^2, k^2 a_1a_2, k^2 a_2^2, k^2 a_2a_3, k^2 a_3^2)}{k^2 a_1^2+ k^2 a_2^2+ k^2 a_3^2} 
=  \frac{ k^2 (a_1^2, a_1 a_2, a_2^2, a_2a_3, a_3^2) }{ k^2 (a_1^2+a_2^2+a_3^2)  }
= \frac{(a_1^2, a_1a_2, a_2^2, a_2a_3, a_3^2)}{a_1^2+a_2^2+a_3^2} $$
Now that we showed that $\phi$ is well-defined, we show that it is injective.

Assume that $\phi$ is not injective, then there are unit vectors $a = [a_1,a_2,a_3]$ and $b=[b_1,b_2,b_3]$ lying on different lines 
such that $\phi([a]) = \phi([b])$ 
In other words $a \in [x:y:z]$  and $b \in [x\prime:y\prime:z\prime]$
$$\phi([a_1,a_2,a_3]) = (a_1^2, a_1a_2, a_2^2, a_2a_3, a_3^2)$$
$$\phi([b_1,b_2,b_3]) = (b_1^2, b_1b_2, b_2^2, b_2b_3, b_3^2)$$
From $\phi([a_1,a_2,a_3]) = \phi([b_1,b_2,b_3])$ we have that $b_1 = \pm a_1$, $b_2 = \pm a_2$, $b_3 = \pm a_3$, and we know that all $b_i$ have the same sign.
Therefore  we either have $b = [a_1, a_2, a_3]$ or $b = [-a_1, -a_2,-a_3]$ which both lie on the line $[x:y:z]$. So we have that $b \in [x:y:z]$ which is a contradiction.
\newline
We proved that $\phi$ is injective, and we know that $\RR \PP ^2$ is compact, so we proceed to proving that $\phi$ is an immersion, 
once we have that we can claim that $\phi$ is an embedding.
\newline

We can use the definition with local charts to prove it. 
Consider the following charts and maps.
$$ u_0 =  \{ [x:y:z], x \neq 0 \} \simeq \RR^2  $$
$$ u_1 =  \{ [x:y:z], y \neq 0 \} \simeq \RR^2  $$
$$ u_2 = \{ [x:y:z], z \neq 0 \} \simeq \RR^2  $$
$$ \psi_0: RP^2 \to \RR^2, \quad
\psi_0([x:y:z]) = (\frac{y}{x}, \frac{z}{x}), \quad
\psi_0^{-1}(s,t) = [1:s,t]
$$
$$ \psi_1: RP^2 \to \RR^2, \quad
\psi_1([x:y:z]) = (\frac{x}{y}, \frac{z}{y}), \quad
\psi_1^{-1}(s,t) = [s:1:t]
$$
$$ \psi_2: RP^2 \to \RR^2, \quad
\psi_2([x:y:z]) = (\frac{x}{z}, \frac{y}{z}), \quad
\psi_2^{-1}(s,t) = [s:t:1]
$$
These local charts cover all the points in $\RR \PP ^2$, to prove that $\phi$ is an immersion
we need to show the following, for all $p \in R^2$
$$ rank(J(\phi \circ \psi_i^{-1}(s,t))) = 2 $$
for $i \in {1,2,3}$.
\newline
We check for $\psi_0$.
$$\phi \circ \psi_0^{-1}(s,t) = [1:s:t] \to (1,s,t,s^2,st,t^2) \frac{1}{1+s^2+t^2} $$
$$ J(\phi \circ \psi_0^{-1}(s,t)) = \frac{1}{(1+s^2+t^2)^2} 
\begin{pmatrix}
-2s & -2t \\
-s^2+t^2+1 & -2st \\
-2st & s^2-t^2+1 \\
2s (t^2+1) & -2 s^2 t \\
t (-s^2 + t^2 +1 ) & s (s^2 - t^2 + 1) \\
-2st^2 & 2t(s^2+1)
\end{pmatrix} 
$$
To see that the rank is always $2$ we can chek the the determinant of minor  $\Delta_{4,6} = 4 s t (s^2 + t^2 + 1)$ which is only zero when $st=0$.
But when both $s =0$ and  $t=0$ equal to zero, the determinant of the minor $\Delta_{2,3} = 1$, and if $\Delta_{2,3}$ is zero only if $s=1$ and $t=0$ or $s=0$ and $t=1$. 
But when that is the case $\Delta_{1,5} \neq 0 $. In conclusion there will always be a $2 \times 2$ minor with non-zero determinant, which means that our matrix has rank $2$.
Similarly we can check that $rank J(\phi_i) = 2$
\newline
\textbf{We can conclude that: The map $\phi$ maps $\RR \PP ^2$ is an immersion, and thus embedding to $\RR^6$}
\newline
\textbf{Think what happens if we consider:}
$$ X = \{ M \; | \; M^2 = M = M^T, trM = k \} \subset Mat_{n \times n} $$
There is an immersion from $G(k,n)$ to $\RR^{n^2}$
$$ \phi: G(k,n) \to X, \quad \phi([U_{n \times k}]) \to A_{n \times n}  $$
$$ \phi([U_{n \times k}]) = U U^T  $$
If bases of $U$ are unit vectors we are good with this formula. If no we need to first normalize each vector in a matrix $U$.
Where one matrix represent a projection of $2$ dimensional plane to a set of two dimensional subspaces, namely:
$$ A \; B  = C  $$
such that $A_{n \times n} \in X$, $ B \in Mat_{n \times k}$ and 
$ C \in [U_{n\times k}] $
$$ \phi G(k,n) \to X, \quad $$
$$ P = A (A^T A)^{-1} A^T $$
\textbf{We can quotient to get grassmann}
\textbf{Normalize and Quotient over G}
\newline
\cite{HuangWG16}
\cite{DBLP}
\bibliographystyle{alpha}
\bibliography{biblio} 
\end{document}          
