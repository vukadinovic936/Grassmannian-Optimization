\documentclass[11pt,a4paper]{report}
\usepackage{graphicx}
\graphicspath{{./images/}}
\usepackage[utf8]{inputenc}
\usepackage{amsmath}
%\usepackage{tabto}
\usepackage{amsfonts}
\usepackage{amssymb}
\usepackage{amsthm}
\usepackage{xcolor}
\usepackage[margin=1in]{geometry}
\linespread{1.25}
\usepackage[colorlinks=true,          % link colors, set to 'false' for print version
            linkcolor=blue,
            citecolor=red,
            urlcolor=blue]{hyperref}
            

%\setlength{\topmargin}{30mm}
%\addtolength{\topmargin}{-1in}
%\addtolength{\topmargin}{-\headsep}
%\addtolength{\topmargin}{-\headheight}
%\addtolength{\topmargin}{-\topskip}

%\setlength{\textheight}{270mm}
%\addtolength{\textheight}{\topskip}
%\addtolength{\textheight}{-\footskip}
%\addtolength{\textheight}{-30pt}

%\setlength{\oddsidemargin}{-1in}
%\addtolength{\oddsidemargin}{20mm}
%\setlength{\evensidemargin}{\oddsidemargin}

%\setlength{\textwidth}{170mm}

\newtheorem{defn}{Definition}[section]
 \newtheorem{thm}{Theorem}[section]
 \newtheorem{Lemma}{Lemma}[section]
 \newtheorem{Claim}{Claim}[section]
 \newtheorem{Prop}{Proposition}[section]
  \theoremstyle{definition}\newtheorem{Ex}{Example}[section]
 \newtheorem{Cor}{Corollary}[section]
 \newtheorem{claim}{Claim}[section]
 \newtheorem{conj}{Conjecture}  

\usepackage {amsfonts,amssymb}
%\usepackage{mathbbol}
\usepackage{latexsym}
\usepackage{mathrsfs}
\input xy
\xyoption{all}

\newcommand {\op}{\mathcal{O}\mathfrak{p}}
\newcommand {\Def}{\textrm{Def}}
\newcommand {\MC} {\textrm{MC}}
\newcommand {\Art}{\textrm{Art}_\CC}
\newcommand {\Kur}{\textrm{Kur}}
\newcommand {\LG} {^LG}
\newcommand{\fart}{\textrm{FArt}_\CC}
\newcommand{\fun}{\textrm{Fun}}
\newcommand{\sets}{\textrm{Sets}}
\newcommand{\tops}{\textrm{Top}}

\newcommand {\BB}{\mathbb{B}}
\newcommand {\CC}{\mathbb{C}}
\newcommand {\FF}{\mathbb{F}}
\newcommand {\KK}{\mathbb{K}}
\newcommand {\MM}{\mathbb{M}}
\newcommand{\NN}{\mathbb{N}}
\newcommand {\PP}{\mathbb{P}}
\newcommand{\QQ}{\mathbb{Q}}
\newcommand {\RR}{\mathbb{R}}
\newcommand {\SSS}{\mathbb{S}}
\newcommand {\VV}{\mathbb{V}}
\newcommand {\HH}{\mathbb{H}}
\newcommand {\WW}{\mathbb{W}}
\newcommand{\YY}{\mathbb{Y}}
\newcommand{\ZZ}{\mathbb{Z}}


\newcommand {\bal}{\boldsymbol{\alpha}}
\newcommand {\bbe}{\boldsymbol{\beta}}
\newcommand {\bga}{\boldsymbol{\gamma}}
\newcommand {\bmu}{\boldsymbol{\mu}}
\newcommand {\bom}{\boldsymbol{\omega}}
\newcommand {\bth}{\boldsymbol{\theta}}
\newcommand {\bph}{\boldsymbol{\phi}}
\newcommand {\bdh}{\boldsymbol{h}}
\newcommand {\bdk}{\boldsymbol{k}}
\newcommand {\bdE}{\boldsymbol{E}}
\newcommand {\bdU}{\boldsymbol{U}}
\newcommand {\bdP}{\boldsymbol{P}}
\newcommand {\ba}{{\bf a}}
\newcommand {\bb}{{\bf b}}
\newcommand {\bc}{{\bf c}}
\newcommand {\bd}{{\bf d}}
\newcommand {\bg}{{\bf g}}
\newcommand {\be}{{\bf e}}
\newcommand {\bdf}{{\bf f}}
\newcommand {\bp}{{\bf p}}
\newcommand {\bq}{{\bf q}}
\newcommand {\bv}{{\bf v}}
\newcommand {\bh}{{\bf h}}
\newcommand {\bk}{{\bf k}}
\newcommand {\br}{{\bf r}}
\newcommand {\bdu}{{\bf u}}
\newcommand {\bdv}{{\bf v}}
\newcommand {\bi}{{\bf i}}
\newcommand {\bj}{{\bf j}}
\newcommand {\bn}{{\bf n}}
\newcommand {\bs}{{\bf s}}
\newcommand {\bt}{{\bf t}}
\newcommand {\bu}{{\bf u}}
\newcommand {\bw}{{\bf w}}
\newcommand {\bx}{{\bf x}}
\newcommand{\by}{{\bf y}}
\newcommand {\bz}{{\bf z}}
\newcommand {\bB}{{\bf B}}
\newcommand{\bD}{{\bf D}}
\newcommand {\bE}{{\bf E}}
\newcommand {\bF}{{\bf F}}
\newcommand {\bG}{{\bf G}}
\newcommand {\bH}{{\bf H}}
\newcommand {\bK}{{\bf K}}
\newcommand {\bL}{{\bf L}}
\newcommand {\bM}{{\bf M}}
\newcommand {\bN}{{\bf N}}
\newcommand {\bO}{{\bf O}}
\newcommand {\bP}{{\bf P}}
\newcommand {\bQ}{{\bf Q}}
\newcommand {\bR}{{\bf R}}
\newcommand {\bT}{{\bf T}}
\newcommand {\bS}{{\bf S}}
\newcommand {\bU}{{\bf U}}
\newcommand {\bV}{{\bf V}}
\newcommand {\bW}{{\bf W}}
\newcommand {\bgamma}{\boldsymbol\gamma}
\newcommand {\bdelta}{\boldsymbol\delta}
\newcommand {\bDelta}{\boldsymbol\Delta}
%\newcommand{\qed}{{\ \bf qed}}

\newcommand {\rroot}{\mathbf{root}}
\newcommand {\coroot}{\mathbf{coroot}}
\newcommand {\weight}{\mathbf{weight}}
\newcommand {\coweight}{\mathbf{coweight}}
 \newcommand{\higgs}{\textrm{Higgs}}
\newcommand{\bun}{\textrm{Bun}}
\newcommand{\rk}{\textrm{rk}}
\newcommand{\ext}{\textrm{Ext}}
% \newcommand {\id}{\mathbb{1}}
% Use \id if using mathbbol instead of amssymb
\newcommand{\range}{\textrm{Range}}
\newcommand{\arccot}{\textrm{arccot}}

\newcommand{\thickslash}{\mathbin{\!\!\pmb{\fatslash}}}


\newcommand{\cA}{\mathcal{A}}
\newcommand{\cB}{\mathcal{B}}
\newcommand{\cC}{\mathcal{C}}
\newcommand{\cD}{\mathcal{D}}
\newcommand{\cE}{\mathcal{E}}
\newcommand{\cF}{\mathcal{F}}
\newcommand{\cG}{\mathcal{G}}
\newcommand{\cH}{\mathcal{H}}
\newcommand{\cI}{\mathcal{I}}
\newcommand{\cJ}{\mathcal{J}}
\newcommand{\cK}{\mathcal{K}}
\newcommand{\cL}{\mathcal{L}}
\newcommand {\cM}{\mathcal{M}}
\newcommand {\cN}{\mathcal{N}}
\newcommand {\cO}{\mathcal{O}}
\newcommand{\cP}{\mathcal{P}}
\newcommand{\cQ}{\mathcal{Q}}
\newcommand{\cR}{\mathcal{R}}
\newcommand{\cS}{\mathcal{S}}
\newcommand{\cT}{\mathcal{T}}
\newcommand{\cU}{\mathcal{U}}
\newcommand{\cV}{\mathcal{V}}
\newcommand{\cW}{\mathcal{W}}
\newcommand{\cX}{\mathcal{X}}
\newcommand{\cY}{\mathcal{Y}}
\newcommand{\cZ}{\mathcal{Z}}

\newcommand{\loc}{\mathcal{L}oc}
\newcommand{\Loc}{\textrm{Loc}}
\newcommand{\cih}{\mathpzc{h}}
\newcommand{\cx}{\mathpzc{x}}
\newcommand{\cy}{\mathpzc{y}}
\newcommand{\ce}{\mathpzc{e}}
\newcommand{\cf}{\mathpzc{f}}
\newcommand{\cl}{\mathpzc{l}}




 




\newcommand{\scA}{\mathscr{A}}
\newcommand{\scB}{\mathscr{B}}
\newcommand{\scC}{\mathscr{C}}
\newcommand{\scD}{\mathscr{D}}
\newcommand{\scE}{\mathscr{E}}
\newcommand{\scF}{\mathscr{F}}
\newcommand{\scG}{\mathscr{G}}
\newcommand{\scH}{\mathscr{H}}
\newcommand{\scI}{\mathscr{I}}
\newcommand{\scJ}{\mathscr{J}}
\newcommand{\scK}{\mathscr{K}}
\newcommand{\scL}{\mathscr{L}}
\newcommand{\scM}{\mathscr{M}}
\newcommand{\scP}{\mathscr{P}}
\newcommand{\scR}{\mathscr{R}}
\newcommand{\scO}{\mathscr{O}}
\newcommand{\scS}{\mathscr{S}}
\newcommand{\scT}{\mathscr{T}}
\newcommand{\scU}{\mathscr{U}}
\newcommand{\scV}{\mathscr{V}}
\newcommand{\scW}{\mathscr{W}}
\newcommand{\scX}{\mathscr{X}}
\newcommand{\scY}{\mathscr{Y}}
\newcommand{\scZ}{\scZ}

\newcommand{\uR}{\underline{\mathbb{R}}}
\newcommand {\uC}{\underline{\mathbb{C}}}


\newcommand{\fh}{\mathfrak{h}}
\newcommand{\fa}{\mathfrak{a}}
\newcommand{\fb}{\mathfrak{b}}
\newcommand{\fc}{\mathfrak{c}}
\newcommand{\fg}{\mathfrak{g}}
\newcommand{\fk}{\mathfrak{k}}
\newcommand{\fl}{\mathfrak{l}}
\newcommand{\fm}{\mathfrak{m}}
\newcommand{\fn}{\mathfrak{n}}
\newcommand{\fo}{\mathfrak{o}}
\newcommand{\fp}{\mathfrak{p}}
\newcommand{\fr}{\mathfrak{r}}
\newcommand{\fs}{\mathfrak{s}}
\newcommand{\fsu}{\mathfrak{su}}
\newcommand{\ft}{\mathfrak{t}}
\newcommand{\slt}{\mathfrak{sl}_2(\CC)}
\newcommand{\sln}{\mathfrak{sl}(n)}
\newcommand{\fsl}{\mathfrak{sl}}
\newcommand{\fu}{\mathfrak{u}}
\newcommand{\fv}{\mathfrak{v}}
\newcommand{\fx}{\mathfrak{x}}
\newcommand{\fy}{\mathfrak{y}}
\newcommand{\fz}{\mathfrak{z}}
\newcommand{\fA}{\mathfrak{A}}
\newcommand{\fB}{\mathfrak{B}}
\newcommand{\fD}{\mathfrak{D}}
\newcommand{\fM}{\mathfrak{M}}
\newcommand{\fR}{\mathfrak{R}}
\newcommand {\fU}{\mathfrak{U}}
\newcommand {\fV}{\mathfrak{V}}
\newcommand {\fW}{\mathfrak{W}}
\newcommand{\fX}{\mathfrak{X}}
\newcommand{\faff}{\mathfrak{aff}}

\newcommand{\Aff}{\textrm{Aff}}

\newcommand{\sym}{\textrm{Sym}}

\newcommand {\dbar}{\overline{\partial}}
\newcommand {\zbar}{\overline{z}}
\newcommand {\zvec}{\underline{z}}
\newcommand {\dzbar}{d\overline{z}}
\newcommand {\Nbar}{\overline{N}}
\newcommand {\Kbar}{\overline{K}}
\newcommand{\diff}{\textrm{Diff}}

%\newcommand {\hom}{\textrm{Hom}}

\newcommand{\mhom}{\textrm{Hom}}
\newcommand {\mend}{\textrm{End}}
\newcommand {\misom}{\textrm{Isom}}
\newcommand {\maut}{\textrm{Aut}}
\newcommand{\pr}{\textrm{pr}}

\newcommand {\sisom}{\underline{Isom}}
\newcommand {\saut}{\underline{Aut}}
\newcommand {\shom}{\textrm{\underline{Hom}}}
\newcommand {\send}{\underline{End} }

\newcommand{\dra}{M^{an}_{DR}(X,G)}
\newcommand{\dr}{M_{DR}(X,G)}

\newcommand {\ad}{\textrm{ad} }
\newcommand{\Ad}{\textrm{Ad}}

\newcommand{\lspan}{\textrm{span}}
\newcommand{\img}{\textrm{Im }}
\newcommand{\spec}{\textrm{Spec }}
\newcommand{\specan}{\textrm{Spec}^{an}}
\newcommand{\gspec}{\underline{\textrm{Spec }}}
\newcommand {\cok}{\textrm{coker}}
\newcommand{\tot}{\textrm{tot }}
\newcommand{\tildel}{\widetilde{\delta}}
\newcommand{\ctimes}{\otimes_\CC}
\newcommand{\sotimes}{\otimes_{\cO_X}}
\newcommand{\pic}{\textrm{Pic}}
\newcommand{\tr}{\textrm{tr }}

\newcommand  {\eps}{\varepsilon}
\newcommand {\kap}{\varkappa}
\newcommand {\io}{\iota}
\newcommand {\fii}{\varphi}

\newcommand{\Higgs}{{\bf Higgs}}
\newcommand{\Bun}{{\bf Bun}}
\newcommand{\gHiggs}{\op{\boldsymbol{\mathcal{H}iggs}}}
\newcommand{\Prym}{{\bf Prym}}
\newcommand{\Jac}{{\bf Jac}}
%\newcommand{\bh}{\boldsymbol{h}}
%\newcommand{\bH}{\boldsymbol{\mathcal{H}}}
\newcommand{\rts}{{\sf root}}
\newcommand{\wts}{{\sf weight}}
\newcommand{\crts}{{\sf coroot}}
\newcommand{\cwts}{{\sf coweight}}
\newcommand{\chr}{{\sf char}}
\newcommand{\cchr}{{\sf cochar}}

\newcommand{\Aut}{\textrm{Aut}}
\newcommand{\Der}{\textrm{Der}}
\newcommand{\spin}{\textrm{Spin}}
\newcommand{\spinc}{\textrm{Spin}^c}
%\newcommand{\U}{\boldsymbol{U(1)}}

\newcommand{\Mat}{\textrm{Mat}}

\newcommand{\hookr}{\hookrightarrow}

%%%%%%%%%%%%%%%%%%%%%%%%%%%%%%%%%%%%%%%%%%%%%%%%%%%%%%%%%%%%%%%%%%%%%%%%%
% Long exact sequence macro
%%%%%%%%%%%%%%%%%%%%%%%%%%%%%%%%%%%%%%%%%%%%%%%%%%%%%%%%%%%%%%%%%%%%%%%%%

\newcommand{\les}[9]{
\xymatrix{
 0 \ar[r] & {#1} \ar[r]  &  {#2} \ar[r]  &  {#3}
\ar@{->}`r/10pt[d] `[l] `^dl[dlll]  `^dr/10pt[dll]    [dll] \\
 &  {#4} \ar[r] & {#5} \ar[r] & {#6}
\ar@{->}`r/10pt[d] `[l] `^dl[dlll]  `^dr/10pt[dll]    [dll] \\
 & {#7} \ar[r]  & {#8} \ar[r] & {#9}
\ar@{->}`r/10pt[d] `[l] `^dl[dlll]  `^dr/10pt[dll]    [dll] \\
 & 0 \ar[r] & \cdots & }
}


%%%%%%%%%%%%%%%%%%%%%%%%%%%%%%%%%%%%%%%%%%%%%%%%%%%%%%%%%%%%%%%%%%%%%%%%%

\newcommand{\lestwo}[9]{
\xymatrix{     
 0 \ar[r] & {#1} \ar[r]  &  {#2} \ar[r]  &  {#3} 
\ar@{->}`r/10pt[d] `[l] `^dl[dlll]  `^dr/10pt[dll]    [dll] \\
 &  {#4} \ar[r] & {#5} \ar[r] & {#6} 
\ar@{->}`r/10pt[d] `[l] `^dl[dlll]  `^dr/10pt[dll]    [dll] \\
 & {#7} \ar[r]  & {#8} \ar[r] & {#9} }
}

%%%%%%%%%%%%%%%%%%%%%%%%%%%%%%%%%%%%%%%%%%%%%%%%%%%%%%%%%%%%%%%%%%%%%%%%%
% Long exact sequence macro
%%%%%%%%%%%%%%%%%%%%%%%%%%%%%%%%%%%%%%%%%%%%%%%%%%%%%%%%%%%%%%%%%%%%%%%%%

\newcommand{\lesthree}[5]{
\xymatrix{     
 0 \ar[r] & {#1} \ar[r]  &  {#2} \ar[r]  &  {#3} 
\ar@{->}`r/10pt[d] `[l] `^dl[dlll]  `^dr/10pt[dll]    [dll] \\
 &  {#4} \ar[r] & {#5} & }
}


%%%%%%%%%%%%%%%%%%%%%%%%%%%%%%%%%%%%%%%%%%%%%%%%%%%%%%%%%%%%%%%%%%%%%%%%%
% Long exact sequence macro
%%%%%%%%%%%%%%%%%%%%%%%%%%%%%%%%%%%%%%%%%%%%%%%%%%%%%%%%%%%%%%%%%%%%%%%%%

\newcommand{\lesfour}[8]{
\xymatrix{     
 0 \ar[r] & {#1} \ar[r]  &  {#2} \ar[r]  &  {#3} 
\ar@{->}`r/10pt[d] `[l] `^dl[dlll]  `^dr/10pt[dll]    [dll] \\
 &  {#4} \ar[r]^-{#8} & {#5} \ar[r] & {#6} 
\ar@{->}`r/10pt[d] `[l] `^dl[dlll]  `^dr/10pt[dll]    [dll] \\
 & {#7} \ar[r]  & \cdots  &  }
}


%


\include{biblio}
\author{Milos Vukadinovic}
\title{G(2,4) is a smooth manifold}
\begin{document}
Let us write $G(2,4) = \{ $2-dimensional (vector) subspaces of $\RR^4 \}$ A plane $V \subseteq \RR^4$ is specified by $4\times2$ matrix $A=[\vec{a}_1,\vec{a}_2] \in Mat_{4\times 2}$ where 
$\{\vec{a}_1, \vec{a}_2\}$ is a basis for V. I.e., given $A \in Mat_{4\times 2}$, $rk A = 2$, we get a 2-plane $V \subseteq \RR^4$ by $V=$span $\{\vec{a}_1,\vec{a}_2\} $. Conversely, given any 2-dim subspace $V\subseteq \RR^4$, there is a $4\times 2$
 matrix A with $rk(A)=2$, from which $V$ is obtained in the above way. Two matrices, $A$ and $B$, determine the same subspace $V \iff \exists g \in GL_2(\RR) $, such that $B = A g$. 
\newline
We thus have the following setup. Let
$$ F(2,4) = \{ A \; | \;  rkA=2 \} \subseteq Mat_{2\times4}(\RR)  \simeq \RR^8 $$
and consider on it the equivalence relation.
$$ B \sim A \text{ if } \exists \; g \in GL_2(\RR), \text{ s.t. } B=Ag $$
We have described a bijection of sets
$$ \equivclass{F(2,4)} \simeq  G(2,4) $$ 
In this problem we argue that $G(2,4)$ can be equipped with a natural smooth structure. 
\newline
\newline
First we show that $F(2,4) \subset Mat_{2 \times 4} (\RR) \simeq \RR^8$ is an open subset
\begin{Lemma}\label{matrank}
    The rank of an $m \times n$ matrix is $r \iff$ some $r \times r$ minor does not vanish,
    and every $(r+1) \times (r+1)$ minor vanishes.
\end{Lemma}
Since we know that $rk(F(2,4)^\complement )<2$, lemma \ref{matrank} tells us all minors
of an arbitary element $ A \in F(2,4)^\complement$ vanish. Let's denote the determinant of each minor of $A$ with $S_i, \; i\in (1,6)$ 
Then consider a continuous map
$ \psi: Mat_{2\times4} \to \RR^6, \quad \psi(M) \to (S_1,S_2,S_3,S_4,S_5,S_6)$.
We can express $F(2,4)^\complement = \psi ^{-1} (0,0,0,0,0,0)$ because all minors vanish (det=0).
A point $(0,0,0,0,0,0) \in \RR^6$ is a closed set, and because continuity preserves the closeness, $F(2,4)^\complement$ is closed in $Mat_{2 \times 4}$,
and since its complement is closed $F(2,4)$ is an open subset of $Mat_{2 \times 4}(\RR)$
\newline
\newline
Next we show that $\simeq$ is an open equivalence relation on $F(2,4)$.
In other words we need to show that the map $\pi: F(2,4) \to \equivclass{F(2,4)}$ is an open map.
\newline
\begin{Lemma} \label{quotientTop}
    A subset of a quotient space is open if and only if its
    preimage under the canonical projection map is open in the original topological space.
\end{Lemma}
Let $U \subset F(2,4)$. Then for every $g \in GL_2(\RR)$ the set $U g = \{ x g | x \in u \}$ is an open subet of $F(2,4)$.
Therefore $\pi^{-1}\pi(u) = \displaystyle \cup \; g u$ is an open in $F(2,4)$ because the union of open sets is open.
And by \ref{quotientTop} $\pi(U) = [U]$ is open in $G(2,4)$. (obviously $\pi$ is a quotient map, and $G(2,4)$ is open in $\RR^8$)
\newline
\newline
\begin{Lemma}
    \label{baset}
    if $\beta = \{ \beta_\alpha \}_\alpha$ is a base for a topology $T$ on a topological space $S$,
    and if $f: S \to X$ is an open map, then the collection $\{ f (\beta_\alpha) \}_\alpha$ is a base for the topology on $X$.
\end{Lemma}
\textbf{Proof:} Let $v$ be an open in $X$ and $y \in V$. Choose $x \in f^{-1}(y)$. 
Since $f^{-1} (V)$ is open there is a basis element $u \in \beta$ s.t. $x \in u \subset f^{-1}(v)$
which implies that $y \in f(u) \subset v$. Since $y$ is arbitary, and $f(U) \subset f(\beta)$ the collection $\{ f (\beta_\alpha) \}_\alpha$ is a base for the topology on $X$.
\newline
\newline
We have that $F(2,4)$ has a second countable base since the subsets keep the properties of the space ($\RR^8$).
Thus by lemma \ref{baset}, we have that the base of $G(2,4)$ is second countable.
\newline
\newline
We show that the graph of the equivalence relation on $F(2,4)$ is a closed subset of $F(2,4) \times F(2,4)$.
I.e $R = \{ (x,y) \in F(2,4) \times F(2,4) \; | \; y = gx \}$ is closed.
We can consider $R$ as a set of matrices. $[A B] = [\vec{a}_1, \vec{a}_2, \vec{b}_1, \vec{b}_2]$ of rank 2.
\ref{matrank} tells us that every $3 \times 3$ minor of an element in $R$ must vanish. So if we take a map:
$$ \psi:  F(2,4) \times F(2,4) \to \RR^{16} $$
$\psi$ takes the determinant of every $3 \times 3$ minor.
$\psi$ is continuous and thus $R = \psi^{-1}(0,0, \dots ,0)$ is closed.
\newline
\begin{center}
      \centering
      \includegraphics[width=0.90\textwidth]{graph_int.png}
\end{center}
From the previous assertion we can conclude that $G(2,4)$ is Hausdorff. 
\newline
That is because $R$ is closed in $F(2,4) \times F(2,4)$, $(F(2,4) \times F(2,4)) \setminus R$ is open.
$\implies \forall (x,y) \in (F(2,4) \times F(2,4)) \setminus R $ there is a basic open set $u \times v$ containing $(x,y)$ s.t. $ (u \times v) \cap R = \emptyset 
\implies \forall x \nsim y \in (F(2,4) \times F(2,4)) \setminus R , \exists$ u around x and v around y s.t. $u \cap v = \emptyset$
Thus for any two points $[x] \neq [y] \in \equivclass{F(2,4)}$ there exist disjoint neighborhood of $x$ and $y$ and $ \equivclass{F(2,4)} $
\newline
\newline
Now that we have Hausdorff property and secound countable basis, we need to prove homeomorphism to claim that $G(2,4)$ is a manifold.
\newline
For $A \in Mat_{4 \times 2} $ denote by $A_{i,j}$ the $2 \times 2$ matrix, formed by the $i-th$ and $j-th$ rows of A $(1 \leq i \le j \leq 4)$. The set
$$ u_{i,j} = \{ A \; | \; det(A_{i,j}) \neq 0 \} \subset F(2,4)$$ is open, because its complement is closed. 
We also have that $\forall g \in GL_2(\RR)$ if $A \in u_{i,j}$ then $ Ag \in u_{i,j}$, because the product of invertible matrices is invertible.
Next, define $v_{i,j} = \equivclass{u_{i,j}} = \pi (u_{i,j}) \subset G(2,4)$.
$v_{i,j}$ is open since the equivalence relation is open.$\; \pi$ is an open map.
Thus $[A]$ has a canonical representative: $A \sim A A_{i,j}^{-1} $
For example if a minor $A_{2,4}$ is invertible we have that
$
A \sim A_{1,3} A_{2,4}^{-1} = 
\begin{pmatrix}
* & * \\
1 & 0 \\
* & * \\
0 & 1
\end{pmatrix}
$
Let $I = \{ 1,2,3,4 \}$ and  define a homeomorphism as follows:
$$ \phi_{i,j}: v_{i,j} \subset G(2,4) \to Mat_{2 \times 2} (\RR) \simeq \RR^4, \quad \phi_{i,j}([A]) = A_{I \setminus \{ i,j \} }A_{i,j}^{-1} $$
It's easy to see that the map is continous since $\phi$ is just a restriciton of $\pi$
\newline
\begin{Ex} \label{ex1}
    Let $[A] \subset v_{3,4} = \begin{bmatrix}
        2 & 6 \\
        1 & 3 \\
        2 & 1 \\
        4 & 3 \\
    \end{bmatrix}$
    $$ A A_{3,4}^{-1} =  \begin{pmatrix} 2 & 6 \\ 1 & 3 \\ 2 & 1 \\ 4 & 3 \\ \end{pmatrix} \begin{pmatrix} \frac{3}{2} & -\frac{1}{2} \\ -2 & 1 \\ \end{pmatrix}
    = \begin{pmatrix} -9 & 5 \\ -\frac{9}{2} & \frac{5}{2} \\ 1 & 0 \\ 0 & 1 \end{pmatrix}
    $$
    the above multiplication is continuous by \ref{quotientTop} and we can exlcude rows $3$ and $4$ so that we get result in $R^4$.
    Then the restriction to $R^4$ is also continuous.
    $$ \phi_{3,4}([A]) = \begin{pmatrix} -9 & 5 \\ -\frac{9}{2} & \frac{5}{2} \end{pmatrix} $$
\end{Ex}
Next the inverse map  $\phi_{3,4}(\beta)^{-1} = \phi_{3,4}(A_{1,2} A_{3,4}^{-1} g) = [A] $ for some matrix $A$, such that $A_{1,2} = \beta$.
But if we pick $g = A_{3,4}$ then 
$ \phi_{3,4}(\beta) =
\begin{bmatrix}
    \beta{1,1} & \beta{1,2} \\ 
    \beta{2,1} & \beta{2,2} \\
    1 & 0 \\ 
    0 & 1 \\
\end{bmatrix}
$
More generally
$$\phi_{i,j}^{-1} : \RR^4 \to v_{i,j} \subset G(2,4) \quad \phi_{i,j}^{-1}(\beta) \to 
\begin{bmatrix}
    \beta \\
    I_{2 \times 2}
\end{bmatrix} = [\alpha]
$$
Such that $\alpha [i:] = \beta[1:]$, $\alpha[j:] = \beta[2:]$ , $\alpha[ (I \setminus \{i,j\})[1] ] = I[1:]$ and
$\alpha[ (I \setminus \{i,j \})[2] ] = I[2:]$
\newline
\begin{Ex}
   $ \phi_{3,4}^{-1} (\alpha) = [A] $ as defined in \ref{ex1}
   $$ \phi_{3,4}^{-1}  \begin{pmatrix} -9 & 5 \\ -\frac{9}{2} & \frac{5}{2} \end{pmatrix} = \begin{bmatrix} -9 & 5 \\ -\frac{9}{2} & \frac{5}{2} \\ 1 & 0 \\ 0 & 1 \end{bmatrix} $$ 
   We can confirm that $\begin{bmatrix} -9 & 5 \\ -\frac{9}{2} & \frac{5}{2} \\ 1 & 0 \\ 0 & 1 \end{bmatrix}$ and $A = \begin{bmatrix}
        2 & 6 \\
        1 & 3 \\
        2 & 1 \\
        4 & 3 \\
    \end{bmatrix}$
    are in the same subspace
    Because if we take $g = \begin{pmatrix} 2 & 1 \\ 4 & 3 \\ \end{pmatrix}$
    then $\alpha g = A$
\end{Ex}
Since $\cup u_{i,j}$ covers $F(2,4)$, $\cup v_{i,j}$ covers $G(2,4)$
Finally, we check transition maps.
$$ \phi_{1,2}([A])^{-1} \to A_{3,4} A_{1,2}^{-1}, \quad \phi_{1,2}^{-1}(u) \to
\begin{pmatrix}
1 & 0 \\
0 & 1 \\
v_{1,1} & v_{1,2} \\
v_{2,1} & v_{2,2}
\end{pmatrix}
$$

$$ \phi_{2,4}([A]) \to A_{1,3} A_{2,4}^{-1}, \quad \phi_{2,4}^{-1}(v) = 
\begin{pmatrix}
v_{1,1} & v_{1,2} \\
1 & 0 \\
v_{2,1} & v_{2,2} \\
0 & 1
\end{pmatrix}
$$
$$ \phi_{2,4} \circ \phi_{1,2}^{-1}(v) = 
\begin{pmatrix}
v_{1,1} & v_{1,2} \\
v_{2,1} & v_{2,2}
\end{pmatrix} \to 
\begin{pmatrix}
    1 & 0 \\
    v_{1,1} & v_{1,2}
\end{pmatrix}
\begin{pmatrix}
    0 & 1 \\
    v_{2,1} & v_{2,2}
\end{pmatrix}
= \begin{pmatrix}
    0 & 1 \\
    v_{1,2} v_{2,1} & v_{1,1} + v_{1,2} v_{2,2}
    \end{pmatrix}
$$
Hence $G(2,4)$ can be equipped with the structure of a 4 dimensional smooth manifold
\newline
Next, we need to generalize the statements above, so that we get that the dimension of grassmannian is$ G(k,n) = n*k-k^2$.
We have as many charts as $k \times k$ minors.
\setcounter{tocdepth}{1} 
\bibliographystyle{alpha}
\bibliography{biblio} 
\end{document}          
